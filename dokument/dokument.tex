\documentclass[a4paper,10pt]{article}
\usepackage[utf8]{inputenc}
\usepackage{graphicx}
\usepackage{xcolor}
\usepackage{amsmath}   
\usepackage{amssymb}
\usepackage{listings}
\usepackage{framed} 
\usepackage{scrtime}
\usepackage[ngerman]{babel} %deutsches Datum
\usepackage[bookmarksopen=true,colorlinks,linkcolor = black]{hyperref}

\usepackage{xcolor}
\definecolor{codecolor}{RGB}{245,222,179}
\definecolor{codegreen}{rgb}{0,0.6,0}
\definecolor{codegray}{rgb}{0.5,0.5,0.5}
\definecolor{codepurple}{rgb}{0.58,0,0.82}
\definecolor{backcolour}{rgb}{0.95,0.95,0.92}
%Stelle Quellcode ein
\lstset{
    backgroundcolor=\color{backcolour},   
    commentstyle=\color{codegreen},
    keywordstyle=\color{magenta},
    numberstyle=\tiny\color{codegray},
    stringstyle=\color{codepurple},
    basicstyle=\footnotesize,    
    language=C++,
    numbers=left,
    captionpos=b,
    frame=shadowbox,
    showspaces=false,
    showstringspaces=false,
    showtabs=false,
    breaklines=true,
    escapeinside={(*•}{•*)},
    numbers=left,
    numbersep=5pt,    
    rulesepcolor=\color{gray},
    framexleftmargin=5mm
}

%Dokument breiter machen
\setlength{\oddsidemargin}{-1cm}
\setlength{\evensidemargin}{-1cm}
\setlength{\textwidth}{16cm}

%Codeumgebung farbig machen
\renewcommand\FrameCommand{\fcolorbox{black}{codecolor}} 


\title{\textbf{XML Technologien\\ Projekt: TierstimmenQuiz}}
\date{\today}
\begin{document}

\maketitle
%#################
%     Begin
%#################
\section{Gruppenmitglieder}
    \begin{itemize}
        \item Jakob Warkotsch
        \item Patrick Schumacher
        \item Peter Schiessl
        \item Kordian Gontarska
        \item Alexander Hinze-Hüttl
        \item Lars Parmakerli
    \end{itemize}

\section{Spielprinzip}
    Das TierstimmenQuiz wird vom Benutzer im Browser gespielt.
    Ihm werden nacheinander 10 Fragen gestellt, die er zu beantworten hat.
    Zu Beginn jeder Frage wird eine Tierstimme abgespielt. Der Benutzer muss 
    diese Tierstimme einem von vier vorgeschlagenen Tieren zuordnen.
    
\section{Aufbau}
        \subsection{Download der Daten}
        Zunächst luden wir die \href{http://offene-naturfuehrer.de/web/Open_Source_Tierstimmen}{Sounddateien der Tierstimmen} runter und speicherten
        Metadaten über die Tierstimmen im \texttt{CSV} Format, dies erledigte ein Ruby Skript.
        
        \subsection{Anreicherung}
            Da \texttt{SPARQL}-Abfragen an einen Trippel Datastore je nach Komplexität des Querys 
            lange dauern können, und weil wir uns unabhängig von Serverausfällen dritter machen
            wollten, entschieden wir uns, unsere Daten zu Beginn mit Informationen
            aus der deutschen DBPedia anzureichern. Ein Python Skript ließt 
            die gegrabbten Daten ein und erstellt eine SPARQL Abfrage, um Daten mit Bildern
            der Tiere und den Wiki Texten anzureichern.
            
       \subsection{Aufbereitung}
           Zunächst wendeten wir eine XSL Transformation an, um die \texttt{CSV} Daten in
           ein von uns definiertes XML Schema zu überführen.\\
           Anschließend mussten wir defekte Datensätze entfernen. Da es zu einigen Tiernamen aus
           dem Tierstimmenarchiv keinen Wikipediaartikel gibt, existieren zu diesen
           Tieren keine Bilder und keine Wiki Texte.
\end{document}